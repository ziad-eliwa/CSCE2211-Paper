\documentclass[conference]{IEEEtran}
\IEEEoverridecommandlockouts
\usepackage{graphicx} % Required for inserting images
\usepackage{cite}
\usepackage{amsmath,amssymb,amsfonts}
\usepackage{algorithmic}
\usepackage{graphicx}
\usepackage{textcomp}
\usepackage{xcolor}

\title{Algorithmic Game Theory and its applications: \\ Stable Matching Problems}
\def\BibTeX{{\rm B\kern-.05em{\sc i\kern-.025em b}\kern-.08em
    T\kern-.1667em\lower.7ex\hbox{E}\kern-.125emX}}

\author{Ziad Eliwa 900246124 - Mostafa Abdelwahed 900243064 \\Ali Mohamed 900246190 - Farouk Youssef 900243941}

\begin{document}

\maketitle

\begin{abstract}
Algorithmic Game Theory (AGT) lies in the intersection between the vast fields of economics and computer science, especially analysis and design of algorithms and theoritical computer science. AGT focuses on designing and modeling algorithmic systems in which the behavior of an entity is influenced by the behaviour of other entities through strategic interactions. In traditional problems in computer science, algorithms expect input to be fixed and reliable. In numerous real world scenarios, like online auctions, internet routing, digital advertising, and resource allocation, inputs come from multiple independent agents who might deliberately deceive data to influence outcomes in their favor. AGT offers tools to study and create systems that function effectively even when faced with such strategic manipulation. This paper discusses Network routing games a fundamental area within AGT that models how self-interested agents (drivers, data packets, autonomous systems) select routes through shared network infrastructure. In these games, each agent's routing decision affects congestion experienced by others, creating complex strategic interactions that can lead to inefficient equilibria. We examine the computational and game-theoretic aspects of routing games, survey key theoretical results including the Price of Anarchy and Braess's Paradox, and present experimental analysis comparing selfish routing behavior against centrally optimized solutions across various network topologies.
\end{abstract}


\section{Introduction}
\section{Problem Definition \& Methodology}
\section{Literature Survey}

\section{Experiment}
\section{Analysis \& Critique}
\section{Conclusion}
\section{Future Work}



\begin{thebibliography}{99}
\bibitem[{\bf}]{dummy1}

\bibitem{bib01}
  
\end{thebibliography}

\bibliographystyle{tjsass} 
\bibliography{references} 
\end{document}
