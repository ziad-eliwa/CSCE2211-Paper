\documentclass[conference]{IEEEtran}
\IEEEoverridecommandlockouts
\usepackage{graphicx}
\usepackage{cite}
\usepackage{amsmath,amssymb,amsfonts}
\usepackage{algorithmic}
\usepackage{graphicx}
\usepackage{textcomp}
\usepackage{xcolor}

\title{Graph Partitioning Algorithms: A Comparative Study}
\def\BibTeX{{\rm B\kern-.05em{\sc i\kern-.025em b}\kern-.08em
    T\kern-.1667em\lower.7ex\hbox{E}\kern-.125emX}}

\author{Ziad Eliwa 900246124 - Mostafa Abdelwahed 900243064 \\Ali Mohamed 900246190 - Farouk Youssef 900243941}

\begin{document}

\maketitle

\begin{abstract}

\end{abstract}

\section{Introduction}
Abstraction is one of the most important ideas of computer science. Computer scientists and researcher across the span of the study and modelling of real-world complex entities have utilized modelling them as \textit{graphs}. In each case, there is an underlying
network that encapsulate the important properties that help us understand these entities and draw meaningful insights from them. In addition to the huge contribution that graphs brought to the advancment in computer science, it brought with its complexities. Due to the enormous data graphs can represent, they often fall in the area of \textit{NP Hard Problems}, starting from the most famous Travelling Salesman Problem (TSP) to graph coloring (k-coloring for k $\geq 3$), Hamiltonian paths and more.
Graph Partitioning stands as one of the most challenging problems in computer science. Graph partitioning is essential for the fields of parallel computing, high performance computing (HPC) and load balancing in networks, VLSI in digital logic design where vertices resemble logical units and edges resemble wires (Science Direct, 2025). This paper goes into depth of some heurisitcs and greedy algorithms that allow local use of the power of graph partitioning.

"This problem is increasing with the emergence of Big Data. Handling tremendous volumes of graph data requires an efficient graph processing system and especially a high-quality graph partitioning approachs to cope with graph application needs." (Sakouhi et al., 2025)
\section{Background}

\section{Problem Definition}
\section{Literature Survey}
\section{Experiment}
\section{Results and Discussion}
\section{Conclusion \& Future Work}


\begin{thebibliography}{99}
\bibitem[{\bf}]{dummy1}

\bibitem{bib01}
\end{thebibliography}

\bibliographystyle{tjsass} 
\bibliography{references} 
\end{document}
